\chapter{Inleiding}
\section{Een koude oorlog}

Waarom noemen we de Koude Oorlog eigenlijk \emph{koud}?

De Koude Oorlog wordt zo genoemd omdat het geen openlijke militaire confrontatie was tussen de Verenigde Staten en de Sovjet-Unie, de twee belangrijkste rivaliserende machten. In plaats daarvan was het een langdurige politieke en ideologische strijd tussen de twee landen die vaak gepaard ging met spionage, propaganda en proxy-oorlogen in andere delen van de wereld.

Hoewel er momenten waren waarop de spanningen tussen de VS en de Sovjet-Unie hoog opliepen en de dreiging van een nucleaire oorlog reëel was, kwam het nooit tot een directe militaire confrontatie tussen de twee grootmachten. Daarom wordt deze periode in de geschiedenis \emph{koud} genoemd, in tegenstelling tot \emph{heet}, wat zou verwijzen naar een openlijk militair conflict.
